%% ------------------------------------------------------------------------- %%
\chapter{Telas OLED Flexíveis}
\label{cap:oled}

A tecnologia OLED é baseada na iluminação através da aplicação de eletricidade em moléculas organicas \cite{HSWOLED}, denominado \bf{eletroluminescência}, como mostra a imagem \ref{fig:oled_early_product}. A \bf{eletroluminescência} foi inicialmente pesquisada por André Bernanose e colegas de trabalho em 1950 na Nancy-Université na França. Eles aplicaram alta tensão (AC) no ar para materias como a \bf{acridina laranja}, depositada e dissolvida em filmes finos de celulose ou celofane. O mecanismo proposto era a excitação direta das moléculas do corante ou a excitação de seus elétrons \cite{WikipediaOLED}.

Em 1960, Martin Pope e colaboradores da Universidade de Nova York desenvolveram eletrodos injetáveis óhmicos de contato a partir de cristais orgânicos. Eles descreveram ainda os requisitos energéticos necessários (funções trabalho) para os eletrodos de contato injetados. Esses contatos são a base da injeção de carga em todos os dispositivos OLED modernos. O grupo de Pope também observou pela primeira vez \bf{eletroluminescência} de corrente contínua (DC) sob vácuo em um único cristal puro de antraceno e em cristais de antraceno embebidos com tetraceno em 1963, usando uma pequena área de eletrodo de prata em 400V. \cite{WikipediaOLED}

\begin{figure}[!h]
  \centering
  \includegraphics[width=.40\textwidth]{oled_early_product} 
  \caption{Uma tela flexível OLED}
  \label{fig:oled_early_product} 
\end{figure}

OLED's são dispositivos solidos compostos composto por finas camadas de moléculas organicas que são capazes de criar eletricidade através da aplicação de eletricidade. OLED's podem prover imagens mais brilhantes, maior nitidez e consumindo menos elergia que suas antecessoras, LCD e LED \cite{HSWOLED}.

O OLED é um dispositivo semi-condutor sólido de 100 a 500 nanometros de espessura, ou seja, aproximadamente 200 vezes mais fino que um fio de cabelo humano. OLED's podem ter de 2 a 3 camadas de material orgânico; sendo que a terceira camada ajuda no transporte de elétron do cátodo para a camada emissiva \ref{fig:camadas_oled} \cite{HSWOLED}.

%% ------------------------------------------------------------------------- %%
\section{Origem}
\label{sec:camadas}

Dr. Ching Tang e Steven Van Slyke são os dois pioneiros da tecnologia OLED - de fato podemos dizer que eles são os inventores da tecnologia OLED em 1987 na Eastman Kodak. Os dois escreveram o artigo \textit{Electroluminescent devices with improved cathodes} para um seminário e este tem sido citado em mais de 5000 publicações. Agora, os dois pioneiros foram incluidos no \textit{hall} da fama dos consumidores de eletronicos \cite{OLEDPioneers}. 

\begin{figure}[!h]
  \centering
  \includegraphics[width=.40\textwidth]{ching_tang} 
  \caption{Criador da tecnologia OLED Ching Tang}
  \label{fig:ching_tang} 
\end{figure}

\begin{figure}[!h]
  \centering
  \includegraphics[width=.40\textwidth]{steven_slyke} 
  \caption{Criador da tecnologia OLED Steven Van Slyke}
  \label{fig:steven_slyke} 
\end{figure}


%% ------------------------------------------------------------------------- %%
\section{Vantagens}
\label{sec:vantagens}


%% ------------------------------------------------------------------------- %%
\section{Composição e Fabricação}
\label{sec:camadas}

\begin{figure}[!h]
  \centering
  \includegraphics[width=.40\textwidth]{camadas_oled} 
  \caption{Componentes OLED}
  \label{fig:camadas_oled} 
\end{figure}


%% ------------------------------------------------------------------------- %%
\subsection{Substrato (\textit{Substrate})}
\label{sec:substrato}

Produzido a partir de plástico transparente, vidro ou folha metálica possui a função de suportar o OLED.


%% ------------------------------------------------------------------------- %%
\subsection{Anódio (\textit{Anode})}
\label{sec:substrato}

Sendo transparente, o anódio remove elétrons, ou seja, adiciona "buracos" de elétrons, enquanto há corrente elétrica no dispositivo.


%% ------------------------------------------------------------------------- %%
\subsection{Camadas Orgânicas (\textit{Organic layers})}
\label{sec:substrato}

Estas camadas são feitas de moléculas organicas ou polímeros, sendo que os tipos de moléculas e polímeros são diferentes entre as camadas.

\bf{Camada Condutiva:} Esta camada é feita de moléculas orgânicas plásticas que transportam os "buracos" do Anódio. Um polímero usado nesta camada é a polianilina.

\bf{Camada Emissiva:} Esta camada é feira de moléculas orgânicas plásticas, diferentes da camada Condutiva, e transporta elétrons do Catódio. É nesta camada onde a luz é gerada. Um dos polímeros usados nesta camada é o polifluoreno. 


%% ------------------------------------------------------------------------- %%
\subsection{Catódio (\textit{Cathode})}
\label{sec:substrato}

Pode ser ou não transparente, a depender do tipo de OLED. Esta camada injeta elétrons enquanto a corrente elétrica flui através do dispositivo. 

O maior trabalho na construção de uma tela OLED é aplicar a camada orgânica ao substrato. Existem três maneiras de se fazer isso:

\begin{enumerate}
	\item \bf{Deposição a vácuo ou evaporação térmica a vácuo:} Em uma câmera de vácuo, as moléculas orgânicas são aquecidas suavemente (evaporadas) e é condensada como finas camadas sobre o substrato arrefecido. Este processo é caro e ineficiente.

	\item \bf{Deposição de vapor orgânico em fase:} Em baixa pressão, em uma câmera com paredes quentes, um gás portador transporta moléculas orgânicas evaporadas para o substrato resfriado, onde se condensam em finas camadas. Utilizar um gás portador aumenta a eficiência e reduz o custo de fabricação de OLED's.

	\item \bf{Impressão a jato de tinta:} Com a tecnologia de jato de tinta, os OLED's são pulverizados sobre substratos como tintas são pulverizadas no papel durante a impressão. A tecnologia jato de tinta reduz muito o custo de produção de OLED e permite que eles sejam impressos em grandes camadas de substrato para grandes telas, como telas de TV de 80 polegadas ou painéis eletrônicos.
\end{enumerate}


%% ------------------------------------------------------------------------- %%
\section{Funcionamento}
\label{sec:funcionamento}

\begin{figure}[!h]
  \centering
  \includegraphics[width=.40\textwidth]{oled_process} 
  \caption{Processo de geração de luz em telas OLED}
  \label{fig:oled_process} 
\end{figure}

Os OLED\'s emitem luz de um modo similar aos LED\'s, através de um processo chamado \bf{eletrofosforescência} descrito a seguir. 

\begin{enumerate}

	\item A bateria ou fonte de alimentação do dispositivo contendo o OLED aplica uma voltagem na tela OLED. 

	\item Uma corrente elétrica é passada do cátodo para o anódio através das camadas orgânicas. O cátodo fornece elétrons à camada emissiva das moléculas orgânicas. O anódio remove elétrons da camada condutora de moléculas orgânicas (isso é o equivalente a dar buracos de elétrons para a camada condutora).

	\item Na fronteira entre a camada emissiva e as camadas condutoras, os elétrons encontram \"buracos\" de elétrons. Quando um elétron encontra um \"buraco\" de elétron, ele preenche o buraco (ele cai no nível de energia do átomo que perdeu um elétron). Quando isso acontece, o elétron fornece energia na forma de um fóton de luz. 

	\item O OLED emite luz. 

	\item A cor da luz depende do tipo de molécula orgânica na camada emissora. Os fabricantes colocam vários tipos de filmes orgânicos no mesmo OLED para fazer displays coloridos.

	\item A intensidade ou brilho da luz depende da quantidade de corrente elétrica aplicada: quanto maior a corrente, maior o brilho da luz.
\end{enumerate}


%% ------------------------------------------------------------------------- %%
\section{Ideias para o futuro}
\label{sec:funcionamento}















%% ------------------------------------------------------------------------- %%
\chapter{Conclus�es}
\label{cap:conclusoes}

O aplicativo foi desenvolvido utilizando como refer�ncia o material de aula da disciplina e estudos sobre a tecnologia Android \cite{Android:2014}. Durante o processo de aprendizado da tecnologia, o grupo enfrentou alguns problemas com respeito � compatibilidade do aplicativo com smartphones de membros do grupo. A vers�o 2.3.6 do Android n�o � compat�vel com as funcionalidades de mapas e GPS que s�o apresentadas no aplicativo da Ouvidoria. Por esse motivo, o processo de desenvolvimento se tornou mais demorado, contudo foi poss�vel implementar todas as funcionalidades solicitadas.

Por meio desse projeto foram exercitadas as principais caracter�sticas de implementa��o com a tecnologia Android e de utiliza��o das principais APIs para uso de mapas e identifica��o da localiza��o do usu�rio via GPS. Al�m disso, o conhecimento sobre o envio de dados via JSON por meio de um aplicativo m�vel foi treinado neste exerc�cio-projeto.


%% ------------------------------------------------------------------------- %%
%Suas considera��es sobre o processo de aprendizado, desenvolvimento e testes do app. 

%Apresenta��o da facilidade/dificuldade para desenvolvimento/debug da aplica��o 
%% ------------------------------------------------------------------------- %%
\chapter{Introdução}
\label{cap:introducao}

Tecnologias móveis estão hoje presentes no cotidiano da maior parte das pessoas. Pesquisas apontam que dentre os meses de Maio e Junho de 2014 o Google bateu o recorde de dispositivos ativos rodando Android, contanto com 1 bilhão e a Apple, no mesmo ano, contava com 800 milhões de dispositivos ativos \cite{ResumoGoogleIOBAM:2014}. 

É inevitável que mais e mais vejamos trabalhos acadêmicos, pesquidas e outros ao redor desta área. Desta forma, este trabalho visa apresenar em mais detalhes os componentes e funcionamento de uma tecnologia emergente sobre telas que vem evoluindo com bastante velocidade e que acredita-se fazer parte da maioria dos dispositivos móveis do futuro, esta tecnologia chama-se Organic Light Emmiting Diode (OLED).

Nos capíulos a seguir faremos um \textit{overview} sobre o passado das tecnologias usadas nas telas de eletroeletrônicos e devices usados no cotidiano como televisores e \textit{smartphones}. abordaremos então sobre a tecnologia OLED e o que já temos no mercado a venda para o consumidor com esta tecnologia. Por último abordaremos sobre algumas ideias que pesquisadores e inovadores pensam para o futuro dos dispositios móveis com esta tecnologia. 


%% ------------------------------------------------------------------------- %%
\section{Motivação}
\label{sec:motivacao}

As tecnologias móveis estão mudando, se tornando cada vez menores, mais resistentes e participando cada vez mais do cotidiano das pessoas como objetos vestíveis. Os componentes usados na construção destes tipos de dispositivos estão se tornando cada vez menores a todo instante. Sabemos que estas tecnologias atuais são muito mais potentes do que supercomputadores que anos atrás ocupavam todo um dormitório.

As tecnologias móveis em particular, estão evolindo com certa intensidade, principalmente em determinadas áreas, e uma das áreas com intensa inovação são sobre as telas. Tela OLED flexível é uma área com desenvolvimento relativamente recente e pode nos apontar muito sobre o futuro das tecnologias móveis \cite{FOLEDDRS:2014}.


%% ------------------------------------------------------------------------- %%
\section{Objetivo}
\label{sec:objetivo}

Este trabalho trata-se de parte integrante do seminário apresentado para a disciplina de Computação Móvel ministrada pelo Professor Alfredo Goldman no primeiro semestre de 2014 no Instituto de Matemática e Estatístca da Universidade de São Paulo.

O objetivo deste trabalho é esclarer o que é, qual a composição, como funciona e quais as aplicabilidades já existente e possíveis futuras sobre a tecnologia OLED. Desta forma, ao final deste trabalho espera-se que leitor tenha um real entendimento sobre esta tecnolgia.

%% ------------------------------------------------------------------------- %%
\section{Organização do Trabalho}
\label{sec:organizacaodotrabalho}

